% !Mode:: "TeX:ACP:Hard"
\documentclass[14pt,hyperref={CJKbookmarks=true}]{beamer}

%\usepackage[space,noindent]{ctex}
\usepackage{mathrsfs}
\usepackage{amsfonts,amssymb}
\usepackage{amsmath}
\usepackage{graphicx}
\usepackage{subcaption}
\usepackage{lmodern}
\usepackage[labelformat=empty,font=scriptsize,skip=0pt,justification=justified,singlelinecheck=false]{caption}
\usepackage{amsthm}
\usepackage{color}
\theoremstyle{plain}
\newtheorem{thm}{Theorem}[section]
\newtheorem{lem}[thm]{Lemma}
\newtheorem{prop}[thm]{Proposition}
\newtheorem*{cor}{Corollary}

\theoremstyle{definition}
\newtheorem{defn}{Definition}[section]
\newtheorem{conj}{Conjecture}[section]
\newtheorem{exmp}{Example}[section]

\theoremstyle{remark}
\newtheorem*{rem}{Remark}

%\captionsetup{font={scriptsize}}
\captionsetup[figure]{name=Fig., labelsep=space}
%remove the icon
\setbeamertemplate{bibliography item}{}
%remove line breaks
\setbeamertemplate{bibliography entry title}{}
\setbeamertemplate{bibliography entry location}{}
\setbeamertemplate{bibliography entry note}{}
%\usetheme{AnnArbor}
\usetheme{CambridgeUS}
%\usetheme{Berlin}
%\usetheme{Dresden}
%\usecolortheme{beaver}
%\setbeamercolor{itemize item}{fg=black!80!black}

\setbeamercolor{normal text}{bg=black!10}
\setbeamerfont{caption}{size=\tiny}
\graphicspath{{image/}}

\begin{document}

%\section{Title}
\title[IACAS]{Particle Detection on Low Contrast Image of Large Aperture Optics}
%\subtitle[副题简称]{论文副题}
\author{Wendong Ding, De Xu, Zhengtao Zhang and Dapeng Zhang}
\institute[]{Institute of Automation Chinese Academy of Sciences\\University of Chinese Academy of Sciences}
\date[]{2016.05.30}
\begin{frame}
\titlepage
\end{frame}


%% Outline page
\begin{frame}
\frametitle{Quick Overview} 
\large\tableofcontents
\end{frame}

%% 
\section{Introduction}
\begin{frame}
\frametitle{Introduction} 
\small
\begin{block}{Particulate contamination deposition}

\begin{enumerate}
\item initiates damage on both bare and coated optical surfaces.
\item decrease of cleanliness level of optics surface.
\item reduces the load capacity of optical systems.
\end{enumerate}
\end{block}

\begin{block}{Online particle detection problem}
\begin{enumerate}
\item challenging problems due to the particle's microsize and optic's large aperture.
\item ease of contaminated by people and external contamination source.
\end{enumerate}

\end{block}
\end{frame}

\begin{frame}
\frametitle{Introduction}{Figure, Research field \& Application}
\begin{columns}[onlytextwidth]
\begin{column}{0.8\textwidth}
\begin{figure}[h]
\centering
\includegraphics[width=0.4\linewidth]{optics-inspection-1.jpg}
\includegraphics[width=0.4\linewidth]{optics-inspection-2.jpg}\\
\includegraphics[width=0.4\linewidth]{optics-inspection-3.jpg}
\includegraphics[width=0.4\linewidth]{optics-inspection-4.jpg}
\caption{
}
\end{figure}
\end{column}
\begin{column}{0.2\textwidth}
\tiny
\begin{tabular}{c|c}
a&b\\\hline
c&d
\end{tabular}\\
\scriptsize
(a) Inspecting and sustaining the cleanliness level of optics\\
(b) Inspecting the particulate contamination on large optics.\\
(c) Inspecting the grass contained neodymium.\\
(d) Inspecting the crystal with size of $\Phi380mm$.
\end{column}
\end{columns}
\end{frame}

\section{Method}

\begin{frame}
\small
\frametitle{Method}{Particle Coarse Detection Algorithm}
\begin{block}{Three steps}
\begin{itemize}
\item A gradient-based edge detection filter is constructed and used as the saliency criterion.
\item Find particle's edge.
\item Generate the whole particle region.
\end{itemize}
\end{block}

\begin{block}{gradient-based edge detection filter}
\begin{equation}\label{GradientImage}
  \mathbf{I}'_s = \frac{1}{Z_s}\sum_{t\in\Omega}g(|\nabla \mathbf{I}_t|;\sigma_s)\mathbf{I}_s
\end{equation}
\scriptsize
$g(x;\sigma)$ : Gaussian function, variance $\sigma^2$. 

$Z_s$ is the normalization term $Z_s = \sum_{t\in\Omega}g(|\nabla\mathbf{I}_t|;\sigma_s)$.
\end{block}
\end{frame}

\begin{frame}
\frametitle{Method}{Particle Coarse Detection Algorithm(cont.)}
\small
\begin{block} {Edge criteria \& Particle Region}
Edge criteria:
\begin{center}
$\mathbf{I}_s - \mathbf{I}'_s \geq \varepsilon_1$
\end{center}
\begin{equation}\label{pixelclass}
\begin{split}
  \mathbf{R}_p = \{s| &\mathbf{I}_s\geq\mathbf{I}_e \enskip  \\
  \&\& \enskip ( & \exists \enskip t \enskip \text{s.t.} \enskip \mathbf{I}_t\geq\mathbf{I}_e \enskip \&\& \enskip t\in U_4(s))\}
\end{split}
\end{equation}
Particle Region
\begin{equation}\label{eq:cluster_criterion}
\begin{split}
&if \quad t_1\in \mathbf{R}^1_p\enskip \&\&  \enskip t_2\in \mathbf{R}^2_p\enskip \&\& \\ &\enskip \enskip \quad t_1,t_2\in U(s) \\
  &then  \quad s \in \mathbf{R}_p
\end{split}
\end{equation}
\end{block}
\end{frame}

\begin{frame}
\frametitle{Method}{Image Alignment}
\begin{columns}[onlytextwidth]
\begin{column}{0.4\textwidth}
\begin{block}{Pose Match}
\scriptsize
\begin{equation}\label{eq:eqRect}
\begin{split}
\mathbf{m}_1 = \lambda_1^{-1}\mathbf{A}[\mathbf{R}_1,\mathbf{t}_1]\mathbf{M}\\
\mathbf{m}_2 = \lambda_2^{-1}\mathbf{A}[\mathbf{R}_2,\mathbf{t}_2]\mathbf{M}
\end{split}
\end{equation}
\begin{equation}
\begin{split}
\mathbf{m}_i = \begin{bmatrix}u_i&v_i&1\end{bmatrix}^T\\
\mathbf{M}=\begin{bmatrix}X&Y&1\end{bmatrix}^T
\end{split}
\end{equation}

\begin{equation}\label{eq:RectHomo}
  \mathbf{m}_2 = \mathbf{H}\mathbf{m}_1
\end{equation}
\begin{equation}\label{eq:GetHomo}
  \mathbf{H} = \lambda^{-1}\lambda_2\mathbf{A}[\mathbf{R}_1,\mathbf{t}_1][\mathbf{R}_2,\mathbf{t}_2]^{-1}\mathbf{A}^{-1}
\end{equation}
\begin{equation}\label{eq:ExtrinsicM}
\mathbf{H}  = \begin{bmatrix}c_{00}&c_{01}&c_{02} \\c_{10}&c_{11}&c_{12}\\ c_{20}&c_{21}&c_{22} \end{bmatrix}
\end{equation}
\end{block}
\end{column}
\begin{column}{0.5\textwidth}
\begin{block}{Solve Homgraph }

\scriptsize


substitute \eqref{eq:ExtrinsicM} into Intrinsic parameter matrix, we can get
\begin{equation}\label{eq:Extrinsicu^*}
u_2  = \frac{c_{00}u_1+c_{01}v_1+c_{02}}{c_{20}u_1+c_{21}v_1+c_{22}}
\end{equation}
\begin{equation}\label{eq:Extrinsicv^*}
v_2  = \frac{c_{10}u_1+c_{11}v_1+c_{12}}{c_{20}u_1+c_{21}v_1+c_{22}}
\end{equation}
\begin{equation}\label{eq:ExtrinsicRew}
\begin{cases}
c'_{00}u_1+c'_{01}v_1+c'_{02} + c'_{20}u_1u_2+c'_{21}v_1u_2=u_2\\
c'_{10}u_1+c'_{11}v_1+c'_{12} + c'_{20}u_1v_2+c'_{21}v_1v_2=v_2
\end{cases}
\end{equation}
where $c'_{ij}=c_{ij}/c_{22}\ (i,j \in{1,2}\&\&i,i\neq2)$. For $n$ given known points, $\mathbf{H}$ can be determined \eqref{eq:ExtrinsicRew}.
\end{block}
\end{column}
\end{columns}
\end{frame}

\begin{frame}
\frametitle{Method}{Image Alignment(cont.)}

\small
To find the matched point pairs, we use Binary Robust Invariant Scalable Keypoints (BRISK) to detect the keypoints and find the matched point pairs.
\begin{itemize}
\item Detects keypoints in octave layers \& layers in-between of the image pyramid.
\item Apply Features from Accelerated Segment Test (FAST) detector.
\item A non-maxima suppression method is performed to find the extreme point.
\item Compute the main direction to have rotate variant.
\item Compute Hamming distance and Match.
\end{itemize}
\end{frame}
\begin{frame}
\frametitle{Method}{Particle Fine Extraction}
\begin{block}{Assumption}
\begin{enumerate}
\item Two particles are concentric or have big degree of overlap.
    \begin{enumerate}
      \item Average intensity around the area changes greatly. This means that a new particle exists in the inspected image.
      \item Average intensity around the area does not change greatly. It means that no new particle exists.
    \end{enumerate}
\item If two particle regions are non-concentric and have small degree of overlap, a new particle condition exists.
\end{enumerate}
\end{block}
\end{frame}

\begin{frame}
\small
\frametitle{Method}{Particle Fine Extraction(cont.)}
\begin{block}{Mean intensity variance}
The mean value inside the curves is $M_p = \frac{1}{N}\sum_{p\in\Gamma}\mathbf{I}(p)$, where $N$ is the pixel number of set $\Gamma$, $\Gamma$ is the set containing all the pixels inside of $\mathbf{B}$. $M_{pp}$ and $M_{pr}$ are the mean intensity
of particle candidates in the inspected image and the reference image, if $M_{pp}-M_{pr}\geq \lambda_2$, then mean intensity
changes greatly.
\end{block}
\begin{block}{Concentric}
Let $\mathbf{P}_i$ be particle candidate set in inspected image and $\mathbf{P}_r$ particle candidate set in reference image, and $D_{ir} = |C_i-C_r|$ where $C_i$ and $C_r$ are the centroids of $\mathbf{P}_i$ and $\mathbf{P}_r$. If $D_{ir} \leq \lambda_1$, the candidates are concentric.
\end{block}
\end{frame}
\begin{frame}
\frametitle{Method}{Particle Fine Extraction(cont.)}
\begin{block}{Overlap definition}
$N_o$ is pixel number of $\Gamma_i\bigcap\Gamma_r$, $N_u$ is pixel number of $\Gamma_i\bigcup\Gamma_r$. The degree of overlap  $D_o=\frac{N_o}{N_u}$. If the degree of overlap of two candidates $D_o \geq \lambda_3$,  two particle candidates are overlap.
\end{block}
\begin{block}{Two cases of overlap}
\begin{enumerate}
\item the contours of two candidates are intersected.
\item one contour fully inside the other contour.
\end{enumerate}
\end{block}
\end{frame}

\begin{frame}{Method}{Particle Classification}
\small
\begin{block}{Feature of Dust and Defects}
\begin{itemize}
\item Digs in defects are point-like in shape as dust particles.
\item particles are usually smooth and round, usually regular in shape.
\item shapes of dust vary greatly.
\end{itemize}
\end{block}

\begin{block}{Feature used for Classification}
\begin{itemize}
\item histogram of gradient.
\item texture. constructed by Gray Level Co-occurrence Matrix (GLCM) from image: Contrast, Correlation, Energy. 
\item Morphological characters. represented by invariant moment. 
\end{itemize}
\end{block}
\end{frame}

\section{Experiment}
\begin{frame}
\frametitle{Experiment}
\begin{block}{The Inspection System}
\begin{enumerate}
\item Experiment system is with a 2.66 GHz processor and 8 GB of memory.
\item Light source.  wavelength at 525nm long with blue light.
\item Used 30 optics images without particle as the reference image and 5 optics images with various sizes of particles. 
\item Used another 8 optics images. for the performance evaluation.
\item The size of the image is $6600\times4400$ pixels.
\end{enumerate}
\end{block}
\end{frame}
\begin{frame}
\frametitle{Experiment}{Inspected Object and Vision System}
\begin{columns}[onlytextwidth]
\begin{column}{0.8\textwidth}
\begin{figure}
\includegraphics[width=0.46\linewidth]{Inspection-System.jpg}
\includegraphics[width=0.5\linewidth]{vision-system.jpg}
\end{figure}
\end{column}
\begin{column}{0.2\textwidth}
\scriptsize
\begin{tabular}{c|c}
a&b
\end{tabular}\\
Real inspection system.  \\
(a) The whole inspection system and the inspected optics.  \\
(b)The vision system composition.
\end{column}
\end{columns}
\end{frame}
\begin{frame}{Particle Coarse detection}
\begin{columns}[onlytextwidth]
\begin{column}{0.9\textwidth}
\begin{figure}
\centering
\includegraphics[width=0.55\linewidth]{particleDetect2D.pdf}
\includegraphics[width=0.45\linewidth]{particleDetect1D.pdf}
\caption{Particle candidate detection result. \\
(a)Magnified part of original optics image. \\
(b)The extracted particle candidates. Three rows of images in (a) and (b) are labeled with horizontal line. \\
(c)Their gray value is shown in (c).} 
\end{figure}
\end{column}
\begin{column}{0.1\textwidth}
\scriptsize
\begin{tabular}{c|c|c}
a&b&c
\end{tabular}
\end{column}
\end{columns}

\end{frame}
\begin{frame}
\frametitle{Experiment}{Particle Fine Extraction}
%On the optics image, we generated various sizes of particles. The diameter of the particle ranged from 1 to 19 pixels. The number distribution over particle size obeyed the normal distribution as .  where $\lambda_4 = 100, \sigma^2_1 = 50$ and $\delta_i$ is the size of particle. The minimum distance $\lambda_1 = 5$, minimum lightness variance $\lambda_2=50$ , and minimum degree of overlap $\lambda_3=0.1$.
\begin{columns}[onlytextwidth]
\begin{column}{0.5\textwidth}
\begin{figure}
\includegraphics[width=\linewidth]{ParticleCountResult.pdf}
\end{figure}
\end{column}
\begin{column}{0.5\textwidth}
\scriptsize
\begin{block}{parameters}

\begin{tabular}{c|c|c}
\hline
Gradient Filter variance& $\sigma_s$ & 5 \\
Edge threshold& $\varepsilon_1$ & 0.03 \\
$\varepsilon$-neighborhood& $\varepsilon_2$&  5\\
Filter window& $\Omega$&5$\times$5\\\hline
\end{tabular}
\end{block}
\begin{block}{Particle generation}
\begin{equation}
\xi=N(\lambda_4/\delta_i,\sigma_1^2)
\end{equation}
\end{block}
\begin{block}{parameters}
\begin{tabular}{c|c|c}
\hline
 particle size distribution mean& $\lambda_4$ & 100\\
 particle size distribution variance&$\sigma^2_1$ &  50\\
 minimum distance  & $\lambda_1$&  5\\
 minimum lightness variance& $\lambda_2$&50\\
 minimum degree of overlap &$\lambda_3$&0.1\\\hline
\end{tabular}
\end{block}
\end{column}
\end{columns}

\end{frame}
\begin{frame}{Experiment}{Particle determination(cont.)}
\begin{columns}[onlytextwidth]
\begin{column}{0.6\textwidth}
\begin{figure}
\includegraphics[width=\linewidth]{sub_result_label2.jpg}
\end{figure}
\end{column}
\begin{column}{0.3\textwidth}
\small
The comparison between the actual particle number and the measured result of the real optics image. It gave the particle number with the error less than 35\%.
\end{column}
\end{columns}
\end{frame}

\begin{frame}
\frametitle{Experiment}{Particle classification}
\begin{columns}[onlytextwidth]
\begin{column}{0.6\textwidth}
\begin{figure}
\includegraphics[width=\linewidth]{classifierOptimization.pdf}
% \includegraphics[width=0.5\linewidth]{paretoChart.pdf}
\end{figure}
\end{column}
\begin{column}{0.3\textwidth}
\small
The result of SVM classifier parameter optimization. When $C=0.1$ and $\gamma=10^{-5}$, the classifier had best performance, then on the test dataset the error was 5\%. 
\end{column}
\end{columns}
\end{frame}
\section{Conclusion}
\begin{frame}{Conclusion}
\small
\begin{enumerate}
\item  The inspected image with actual particle has low contrast.
\item  Particle detection algorithm preformed only on the inspected image cannot realize accurately identifying.
\item A particle coarse detection method was proposed to  detect all possible particles, then the reference image was utilized to fine extract the actual particles.
\item It demonstrated a good ability of detecting particles on low contract image of large aperture optics which is implemented by several steps on the inspected image.
\end{enumerate}
\end{frame}
%\section{Thanks}
\begin{frame}
%\frametitle{Thank you}
\Huge
\begin{center}
Thank you
\end{center}


%\begin{block}{Collaborators}
%De Xu, Zhengtao Zhang and Dapeng Zhang
%\end{block}
%\begin{block}{Foundation}
% National Nature Science Foundation under Grant 61473293, 61227804 and 61303177
%\end{block}
%\begin{block}{For more information about the research itself:}
%De Xu, Zhengtao Zhang and Dapeng Zhang
%Institute of Automation, Chinese Academy of Sciences, Beijing 100190
%E-mail: dingwendong2013@ia.ac.cn
%\end{block}
\end{frame}
\end{document}
